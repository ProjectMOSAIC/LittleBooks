\section{Numerical summaries}

\myindex{categorical variables}
\myindex{contingency tables}
\myindex{tables}

\DiggingDeeper{The \emph{Start Teaching with R} companion book introduces the formula notation used throughout this book. See also \emph{Start Teaching with R} for the connections to statistical modeling.}

The \function{tally()} function can be used to calculate
counts, percentages and proportions for a categorical variable.

\Rindex{tally()}%
\Rindex{margins option}%
\begin{knitrout}
\definecolor{shadecolor}{rgb}{0.969, 0.969, 0.969}\color{fgcolor}\begin{kframe}
\begin{alltt}
\hlkwd{tally}\hlstd{(}\hlopt{~} \hlstd{homeless,} \hlkwc{data}\hlstd{=HELPrct)}
\end{alltt}


{\ttfamily\noindent\bfseries\color{errorcolor}{\#\# Error in tally(\textasciitilde{}homeless, data = HELPrct): could not find function "{}tally"{}}}\begin{alltt}
\hlkwd{tally}\hlstd{(}\hlopt{~} \hlstd{homeless,} \hlkwc{margins}\hlstd{=}\hlnum{TRUE}\hlstd{,} \hlkwc{data}\hlstd{=HELPrct)}
\end{alltt}


{\ttfamily\noindent\bfseries\color{errorcolor}{\#\# Error in tally(\textasciitilde{}homeless, margins = TRUE, data = HELPrct): could not find function "{}tally"{}}}\begin{alltt}
\hlkwd{tally}\hlstd{(}\hlopt{~} \hlstd{homeless,} \hlkwc{format}\hlstd{=}\hlstr{"percent"}\hlstd{,} \hlkwc{data}\hlstd{=HELPrct)}
\end{alltt}


{\ttfamily\noindent\bfseries\color{errorcolor}{\#\# Error in tally(\textasciitilde{}homeless, format = "{}percent"{}, data = HELPrct): could not find function "{}tally"{}}}\begin{alltt}
\hlkwd{tally}\hlstd{(}\hlopt{~} \hlstd{homeless,} \hlkwc{format}\hlstd{=}\hlstr{"proportion"}\hlstd{,} \hlkwc{data}\hlstd{=HELPrct)}
\end{alltt}


{\ttfamily\noindent\bfseries\color{errorcolor}{\#\# Error in tally(\textasciitilde{}homeless, format = "{}proportion"{}, data = HELPrct): could not find function "{}tally"{}}}\end{kframe}
\end{knitrout}

\section{The binomial test}

\myindex{binomial test}%
\Rindex{binom.test()}%
An exact confidence interval for a proportion (as well as a test of the null 
hypothesis that the population proportion is equal to a particular value [by default 0.5]) can be calculated
using the \function{binom.test()} function.
The standard \function{binom.test()} requires us to tabulate.
\begin{knitrout}
\definecolor{shadecolor}{rgb}{0.969, 0.969, 0.969}\color{fgcolor}\begin{kframe}
\begin{alltt}
\hlkwd{binom.test}\hlstd{(}\hlnum{209}\hlstd{,} \hlnum{209} \hlopt{+} \hlnum{244}\hlstd{)}
\end{alltt}
\begin{verbatim}
## 
## 	Exact binomial test
## 
## data:  209 and 209 + 244
## number of successes = 209, number of trials = 453, p-value =
## 0.1101
## alternative hypothesis: true probability of success is not equal to 0.5
## 95 percent confidence interval:
##  0.4147418 0.5085030
## sample estimates:
## probability of success 
##              0.4613687
\end{verbatim}
\end{kframe}
\end{knitrout}
The \pkg{mosaic} package provides a formula interface that avoids the need to pre-tally
the data.
\begin{knitrout}
\definecolor{shadecolor}{rgb}{0.969, 0.969, 0.969}\color{fgcolor}\begin{kframe}
\begin{alltt}
\hlstd{result} \hlkwb{<-} \hlkwd{binom.test}\hlstd{(}\hlopt{~} \hlstd{(homeless}\hlopt{==}\hlstr{"homeless"}\hlstd{),} \hlkwc{data}\hlstd{=HELPrct)}
\end{alltt}


{\ttfamily\noindent\bfseries\color{errorcolor}{\#\# Error in binom.test(\textasciitilde{}(homeless == "{}homeless"{}), data = HELPrct): unused argument (data = HELPrct)}}\begin{alltt}
\hlstd{result}
\end{alltt}


{\ttfamily\noindent\bfseries\color{errorcolor}{\#\# Error in eval(expr, envir, enclos): object 'result' not found}}\end{kframe}
\end{knitrout}

As is generally the case with commands of this sort, 
there are a number of useful quantities available from 
the object returned by the function.  
\begin{knitrout}
\definecolor{shadecolor}{rgb}{0.969, 0.969, 0.969}\color{fgcolor}\begin{kframe}
\begin{alltt}
\hlkwd{names}\hlstd{(result)}
\end{alltt}


{\ttfamily\noindent\bfseries\color{errorcolor}{\#\# Error in eval(expr, envir, enclos): object 'result' not found}}\end{kframe}
\end{knitrout}
These can be extracted using the {\tt \$} operator or an extractor function.
For example, the user can extract the confidence interval or p-value.
\Rindex{confint()}%
\Rindex{pval()}%
\Rindex{print()}%
\begin{knitrout}
\definecolor{shadecolor}{rgb}{0.969, 0.969, 0.969}\color{fgcolor}\begin{kframe}
\begin{alltt}
\hlstd{result}\hlopt{$}\hlstd{statistic}
\end{alltt}


{\ttfamily\noindent\bfseries\color{errorcolor}{\#\# Error in eval(expr, envir, enclos): object 'result' not found}}\begin{alltt}
\hlkwd{confint}\hlstd{(result)}
\end{alltt}


{\ttfamily\noindent\bfseries\color{errorcolor}{\#\# Error in confint(result): object 'result' not found}}\begin{alltt}
\hlkwd{pval}\hlstd{(result)}
\end{alltt}


{\ttfamily\noindent\bfseries\color{errorcolor}{\#\# Error in pval(result): could not find function "{}pval"{}}}\end{kframe}
\end{knitrout}
\DiggingDeeper{Most of the objects in \R\ have a \function{print()}
method.  So when we get \code{result}, what we are seeing displayed in the console is
\code{print(result)}.  There may be a good deal of additional information
lurking inside the object itself.  

In some situations, such as graphics, the object is returned \emph{invisibly}, so nothing prints.  That avoids your having to look at a long printout not intended for human consumption. You can still assign the returned object to a variable and process it later, even if nothing shows up on the screen.  This is sometimes helpful for \pkg{lattice} graphics functions.}%


\section{The proportion test}

A similar interval and test can be calculated using the function \function{prop.test()}.
\Rindex{prop.test()}%
\Rindex{correct option}%
Here is a count of the number of people at each of the two levels of \variable{homeless}

\begin{knitrout}
\definecolor{shadecolor}{rgb}{0.969, 0.969, 0.969}\color{fgcolor}\begin{kframe}
\begin{alltt}
\hlkwd{tally}\hlstd{(}\hlopt{~} \hlstd{homeless,} \hlkwc{data}\hlstd{=HELPrct)}
\end{alltt}


{\ttfamily\noindent\bfseries\color{errorcolor}{\#\# Error in tally(\textasciitilde{}homeless, data = HELPrct): could not find function "{}tally"{}}}\end{kframe}
\end{knitrout}

The \function{prop.test} function will carry out the calculations of the proportion test and report the result.

\hfill

\begin{knitrout}
\definecolor{shadecolor}{rgb}{0.969, 0.969, 0.969}\color{fgcolor}\begin{kframe}
\begin{alltt}
\hlkwd{prop.test}\hlstd{(}\hlopt{~} \hlstd{(homeless}\hlopt{==}\hlstr{"homeless"}\hlstd{),} \hlkwc{correct}\hlstd{=}\hlnum{FALSE}\hlstd{,} \hlkwc{data}\hlstd{=HELPrct)}
\end{alltt}


{\ttfamily\noindent\bfseries\color{errorcolor}{\#\# Error in prop.test(\textasciitilde{}(homeless == "{}homeless"{}), correct = FALSE, data = HELPrct): unused argument (data = HELPrct)}}\end{kframe}
\end{knitrout}
In this statement, prop.test is examining the \variable{homeless} variable in the same way that \function{tally} would. \Pointer{We write \code{homeless=="homeless"} to define unambiguously which proportion we are considering. We could also have written \code{homeless=="housed"}. }
\function{prop.test} can also work directly with numerical counts, the way \function{binom.test()} does.
\InstructorNote{\function{prop.test()} calculates a Chi-squared statistic.
Most introductory texts use a $z$-statistic.  They are mathematically equivalent in terms of inferential statements, but you may need to address the discrepancy with your students.}%
\begin{knitrout}
\definecolor{shadecolor}{rgb}{0.969, 0.969, 0.969}\color{fgcolor}\begin{kframe}
\begin{alltt}
\hlkwd{prop.test}\hlstd{(}\hlnum{209}\hlstd{,} \hlnum{209} \hlopt{+} \hlnum{244}\hlstd{,} \hlkwc{correct}\hlstd{=}\hlnum{FALSE}\hlstd{)}
\end{alltt}
\begin{verbatim}
## 
## 	1-sample proportions test without continuity correction
## 
## data:  209 out of 209 + 244, null probability 0.5
## X-squared = 2.7042, df = 1, p-value = 0.1001
## alternative hypothesis: true p is not equal to 0.5
## 95 percent confidence interval:
##  0.4159798 0.5074072
## sample estimates:
##         p 
## 0.4613687
\end{verbatim}
\end{kframe}
\end{knitrout}

\section{Goodness of fit tests}

A variety of goodness of fit tests can be calculated against a reference  distribution.  For the HELP data, we could test the null hypothesis that there is an equal proportion of subjects in each substance abuse group back in the original populations.


\begin{knitrout}
\definecolor{shadecolor}{rgb}{0.969, 0.969, 0.969}\color{fgcolor}\begin{kframe}
\begin{alltt}
\hlkwd{tally}\hlstd{(}\hlopt{~} \hlstd{substance,} \hlkwc{format}\hlstd{=}\hlstr{"percent"}\hlstd{,} \hlkwc{data}\hlstd{=HELPrct)}
\end{alltt}


{\ttfamily\noindent\bfseries\color{errorcolor}{\#\# Error in tally(\textasciitilde{}substance, format = "{}percent"{}, data = HELPrct): could not find function "{}tally"{}}}\begin{alltt}
\hlstd{observed} \hlkwb{<-} \hlkwd{tally}\hlstd{(}\hlopt{~} \hlstd{substance,} \hlkwc{data}\hlstd{=HELPrct)}
\end{alltt}


{\ttfamily\noindent\bfseries\color{errorcolor}{\#\# Error in tally(\textasciitilde{}substance, data = HELPrct): could not find function "{}tally"{}}}\begin{alltt}
\hlstd{observed}
\end{alltt}


{\ttfamily\noindent\bfseries\color{errorcolor}{\#\# Error in eval(expr, envir, enclos): object 'observed' not found}}\end{kframe}
\end{knitrout}

\Caution[-1cm]{In addition to the \option{format} option, there is an option \option{margins} to include marginal totals in the table. The default in \function{tally} is \option{margins=FALSE}. Try it out!}
\Rindex{chisq.test()}%
\begin{knitrout}
\definecolor{shadecolor}{rgb}{0.969, 0.969, 0.969}\color{fgcolor}\begin{kframe}
\begin{alltt}
\hlstd{p} \hlkwb{<-} \hlkwd{c}\hlstd{(}\hlnum{1}\hlopt{/}\hlnum{3}\hlstd{,} \hlnum{1}\hlopt{/}\hlnum{3}\hlstd{,} \hlnum{1}\hlopt{/}\hlnum{3}\hlstd{)}   \hlcom{# equivalent to rep(1/3, 3)}
\hlkwd{chisq.test}\hlstd{(observed,} \hlkwc{p}\hlstd{=p)}
\end{alltt}


{\ttfamily\noindent\bfseries\color{errorcolor}{\#\# Error in is.data.frame(x): object 'observed' not found}}\begin{alltt}
\hlstd{total} \hlkwb{<-} \hlkwd{sum}\hlstd{(observed); total}
\end{alltt}


{\ttfamily\noindent\bfseries\color{errorcolor}{\#\# Error in eval(expr, envir, enclos): object 'observed' not found}}

{\ttfamily\noindent\bfseries\color{errorcolor}{\#\# Error in eval(expr, envir, enclos): object 'total' not found}}\begin{alltt}
\hlstd{expected} \hlkwb{<-} \hlstd{total}\hlopt{*}\hlstd{p; expected}
\end{alltt}


{\ttfamily\noindent\bfseries\color{errorcolor}{\#\# Error in eval(expr, envir, enclos): object 'total' not found}}

{\ttfamily\noindent\bfseries\color{errorcolor}{\#\# Error in eval(expr, envir, enclos): object 'expected' not found}}\end{kframe}
\end{knitrout}

We can also calculate the $\chi^2$ statistic manually, as a function of observed and expected values.

\Rindex{sum()}%
\Rindex{pchisq()}%
\begin{knitrout}
\definecolor{shadecolor}{rgb}{0.969, 0.969, 0.969}\color{fgcolor}\begin{kframe}
\begin{alltt}
\hlstd{chisq} \hlkwb{<-} \hlkwd{sum}\hlstd{((observed} \hlopt{-} \hlstd{expected)}\hlopt{^}\hlnum{2}\hlopt{/}\hlstd{(expected)); chisq}
\end{alltt}


{\ttfamily\noindent\bfseries\color{errorcolor}{\#\# Error in eval(expr, envir, enclos): object 'observed' not found}}

{\ttfamily\noindent\bfseries\color{errorcolor}{\#\# Error in eval(expr, envir, enclos): object 'chisq' not found}}\begin{alltt}
\hlnum{1} \hlopt{-} \hlkwd{pchisq}\hlstd{(chisq,} \hlkwc{df}\hlstd{=}\hlnum{2}\hlstd{)}
\end{alltt}


{\ttfamily\noindent\bfseries\color{errorcolor}{\#\# Error in pchisq(chisq, df = 2): object 'chisq' not found}}\end{kframe}
\end{knitrout}
\FoodForThought[-2cm]{The \function{pchisq} function calculates the probability that a $\chi^2$ random variable with \function{df} degrees is freedom is less than or equal to a given value.  Here we calculate the complement to find the area to the right of the observed Chi-square statistic.}%

It may be helpful to consult a graph of the statistic, where the shaded area represents the value to the right of the observed value.

\Rindex{plotDist()}%
\begin{knitrout}
\definecolor{shadecolor}{rgb}{0.969, 0.969, 0.969}\color{fgcolor}\begin{kframe}
\begin{alltt}
\hlkwd{plotDist}\hlstd{(}\hlstr{"chisq"}\hlstd{,} \hlkwc{df}\hlstd{=}\hlnum{2}\hlstd{,} \hlkwc{groups} \hlstd{= x} \hlopt{>} \hlnum{9.31}\hlstd{,} \hlkwc{type}\hlstd{=}\hlstr{"h"}\hlstd{)}
\end{alltt}


{\ttfamily\noindent\bfseries\color{errorcolor}{\#\# Error in plotDist("{}chisq"{}, df = 2, groups = x > 9.31, type = "{}h"{}): could not find function "{}plotDist"{}}}\end{kframe}
\end{knitrout}


Alternatively, the \pkg{mosaic} package provides a version of \function{chisq.test()} with more verbose output.
\Rindex{xchisq.test()}%
\begin{knitrout}
\definecolor{shadecolor}{rgb}{0.969, 0.969, 0.969}\color{fgcolor}\begin{kframe}
\begin{alltt}
\hlkwd{xchisq.test}\hlstd{(observed,} \hlkwc{p}\hlstd{=p)}
\end{alltt}


{\ttfamily\noindent\bfseries\color{errorcolor}{\#\# Error in xchisq.test(observed, p = p): could not find function "{}xchisq.test"{}}}\end{kframe}
\end{knitrout}
\FoodForThought[-1.5cm]{\code{x} in \function{xchisq.test} stands for eXtra.}

\FoodForThought{Objects in the workspace are listed in the {\sc Environment} tab in \RStudio.  If you want to clean up that listing, remove objects that are no longer needed with \function{rm}.}
\begin{knitrout}
\definecolor{shadecolor}{rgb}{0.969, 0.969, 0.969}\color{fgcolor}\begin{kframe}
\begin{alltt}
\hlcom{# clean up variables no longer needed}
\hlkwd{rm}\hlstd{(observed, p, total, chisq)}
\end{alltt}


{\ttfamily\noindent\color{warningcolor}{\#\# Warning in rm(observed, p, total, chisq): object 'observed' not found}}

{\ttfamily\noindent\color{warningcolor}{\#\# Warning in rm(observed, p, total, chisq): object 'total' not found}}

{\ttfamily\noindent\color{warningcolor}{\#\# Warning in rm(observed, p, total, chisq): object 'chisq' not found}}\end{kframe}
\end{knitrout}
